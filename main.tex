%TC:ignore
\documentclass[%
    fontsize=12pt,
    numbers=noenddot,
    paper=a4,
    toc=bib,
]{scrartcl}

% Language using polyglossia
\usepackage{polyglossia}
\setmainlanguage[variant=british,ordinalmonthday=false]{english}

% biblatex
\usepackage[%
    backend=biber,
    doi=true,
    language=british,
    maxbibnames=9,
    maxcitenames=2,
    mincrossrefs=3,
    natbib=true,
    sortcites,
    style=unified,
    useprefix=true,
    ]{biblatex}

\assignrefcontextentries[]{*}

% remove "p(p)."
\DeclareFieldFormat{postnote}{#1} % no postnote prefix in "normal" citation commands

% no shorthands

\DeclareFieldInputHandler{shorthand}{\def\NewValue{}}

% New citeposs command, more flexible

\usepackage{xifthen}

% use pubstate

\DeclareSourcemap{%
    \maps[datatype=bibtex]{%
        \map[overwrite=true]{% Notice the overwrite: replace one field with another
            \step[fieldsource=pubstate]
            \step[fieldset=year, origfieldval]
        }
    }
}

%% but don't print pubstate in the bibliography apart from the year

\AtEveryBibitem{%
    \clearfield{pubstate}%
}

%% add "to appear" as a proper pubstate, including capitalisation

\NewBibliographyString{toappear}
\DefineBibliographyStrings{english}{%
  toappear = {to appear},
}


%% Fonts, languages

\usepackage[warnings-off={mathtools-colon, mathtools-overbracket}]{unicode-math}

% kxetex
\usepackage{fontspec}
\defaultfontfeatures{Mapping=tex-text}

\usepackage[sb]{libertinus-otf}
\usepackage[scale=MatchLowercase]{FiraMono}

% Nicer tables
\usepackage{booktabs}
\usepackage[table]{xcolor}
\usepackage{colortbl}
\usepackage{microtype}
\usepackage{tabularx}

\usepackage[unicode,hidelinks]{hyperref}
\usepackage[normalem]{ulem} % for strikethrough \sout

%% captions (KOMA-script)
\setkomafont{caption}{\small}
\setkomafont{captionlabel}{\small\bfseries}
\renewcommand{\captionformat}{\quad}

%% Headers and footers
\usepackage{textcase}
\usepackage[automark,markcase=lower]{scrlayer-scrpage}
\pagestyle{scrheadings}
\setkomafont{pagenumber}{\footnotesize}%\sffamily}
\setkomafont{pageheadfoot}%
    {\small\addfontfeature{LetterSpace=10,Numbers=OldStyle}\scshape\sffamily}
\clearpairofpagestyles{}
\cfoot[\pagemark]{\pagemark}
\rohead{\MakeTextLowercase{\emph{\shorttitle}}}
\lohead{\MakeTextLowercase{\authorlast}}

%% Titles and headings (KOMA-Script)

% Titlepage
\setkomafont{author}{\large\sffamily}

% TOC
\setcounter{tocdepth}{2}
\setkomafont{partentry}{\large\sffamily\bfseries}
\KOMAoptions{headings=small}

% Chapter/section headings
\setcounter{secnumdepth}{3}

% Linguistics
\usepackage{langsci-gb4e}
\newcommand{\judgement}[1]{\makebox[0pt][r]{#1}}

%% glossary of abbreviations
\usepackage{glossaries}
\usepackage{glossary-inline}
\usepackage{leipzig}
\makeglossaries{}

\usepackage{cleveref}
\usepackage{enumitem}

%% Line breaks
\widowpenalty=10000
\clubpenalty=10000

% useful 
\newcommand{\email}[1]{\href{mailto:#1}{#1}}
\usepackage[autostyle=true,english=american]{csquotes}
\renewcommand*{\mkccitation}[1]{ (#1)}

% title page
\title{\fulltitle}
\date{}

%TC:endignore


% Add your bibliography here
\addbibresource{main.bib}

%% Specify paper metadata ====

% full title of paper
\newcommand{\fulltitle}{GLOW proceedings template}
% short title of paper for header
\newcommand{\shorttitle}{GLOW proceedings template}
% last name of author(s); for multiple use X & Y; X, Y, & Z; X et al.
\newcommand{\authorlast}{Bárány}

%% Specify paper metadata ====

% Specify author(s)
\author{%
    András Bárány \orcidlink{0000-0003-2356-4338}\\
    University of Edinburgh\\
    \email{andras.barany@ed.ac.uk}
}

% Specify your own packages here (keep this to a minimum!)
\usepackage[newfloat]{minted}
\newacronym{wco}{WCO}{weak crossover}

\usepackage{tikz}
\usetikzlibrary{arrows.meta}
\usepackage[linguistics]{forest}

\begin{document}

\maketitle
\thispagestyle{titlepage}
\pagestyle{scrheadings}

\begin{abstract}
    \noindent\textbf{Abstract:} This document decribes the GLOW proceedings
    style guide and provides some guidance for using it in \LaTeX{}. You can
    simply make a copy of the file \texttt{main.tex} and use it to write your
    GLOW proceedings paper.\\

    \noindent\textbf{Keywords:} \emph{keyword 1}, \emph{keyword 2}, \emph{keyword 3}
\end{abstract}

\section{This template}

If you are reading this file, you have downloaded the GLOW proceedings template
or you have accessed it on overleaf. This template consists of three files:

\begin{itemize}
    \item \texttt{glow-proceedings-preamble.tex}
    \item \texttt{main.tex}
    \item \texttt{main.bib}
\end{itemize}

You can ignore \texttt{glow-proceedings-preamble.tex} and simply work with
\texttt{main.tex} to write your paper and \texttt{main.bib} to include your
bibliography. If you require any additional packages, add them to
\texttt{main.tex} just before the beginning of the document.

This file describes the template and serves as a style sheet for GLOW
proceedings. Please follow the guidelines stated here otherwise we might not be
able to accept your submission. The editors \textbf{will not correct
submissions} but only check them for adherence to the GLOW style. Authors will
be given an opportunity to revise their submission, but non-adherence to the
style sheet may lead to a contribution being excluded from the proceedings.

\section{Style guidelines}

It's easiest to follow the style guidelines if you use the \LaTeX{} template:
you can download it or use it directly on overleaf. We very much welcome
submissions in \LaTeX{} as they make editing much easier.

Here are a few requirements for the proceedings:

\begin{itemize}
    \item Paper size: A4
    \item Typefaces (\enquote{fonts}): \textbf{Libertinus Serif} for text,
        \textbf{Libertinus Sans} for the title, author information, and
        headings\footnote{Why these typefaces? They are freely available
        open-source typefaces with very good support for Unicode characters.
        If you're using a recent version of \LaTeX{}, you'll have them
        installed already. If not, you can download them from
        \faExternalLink*{}~\url{https://github.com/alerque/libertinus/releases}.}
    \item Each paper has an abstract (below the title)
    \item Each paper has up to five keywords (below the abstract)
    \item For citations and the bibliography, follow the
    \faExternalLink*{}~\href{https://clas.wayne.edu/linguistics/resources/style}{Unified
    Style Sheet for Linguistics}; if you use \LaTeX{}, you're set with this
    template (see~\Cref{sec:bibliography})
\end{itemize}

\subsection{Headings}

Headings are set in \textsf{\textbf{Libertinus Sans}} (12pt, bold, upright).
Sections (N), subsections (N.M), and subsubsections (N.M.O) are numbered. An
optional list of abbreviations and the references have unnumbered headings.
Use sentence case in headings, that is capitalise the first word as well as
words which are generally capitalised such as proper names, but nothing else.

\subsection{Headers and footers}

In headers, the left side features the last name of one or multiple authors.
Separate author last names using \enquote{\&} and use \enquote{X et al.} for
more than two authors, in small capitals. The right side features the paper's
(short) title, in italic small capitals.
%
The footer consists of the centered current page number.

\subsection{Bibliography}

The template uses the \texttt{unified} citation and bibliography style in
\texttt{biblatex} (implementing the
\faExternalLink*{}~\href{https://clas.wayne.edu/linguistics/resources/style}{Unified
    Style Sheet for Linguistics}).\footnote{You can also find a CLS file for use
    with Zotero and other reference managers here:
\faExternalLink*{}~\url{https://github.com/citation-style-language/styles/blob/master/unified-style-sheet-for-linguistics.csl}}
See also~\Cref{sec:bibliography} on more details regarding citations in \LaTeX.

\subsection{Examples}

Number your examples. The \LaTeX{} template uses the package
\faExternalLink*{}
\href{https://ctan.mines-albi.fr/macros/xetex/latex/langsci/documentation/langsci-gb4.pdf}{\texttt{langsci-gb4e}}.
If you know what you're doing, you can use another package.

Specify the language of examples, either in the text, or as part of the example,
as in~\eqref{ex:hungarian}, and provide sources where relevant. Provide glosses
for examples where relevant and use the
\faExternalLink*{}~\href{https://www.eva.mpg.de/lingua/pdf/Glossing-Rules.pdf}{Leipzig
Glossing Rules}. If you use the \texttt{leipzig} package
(see~\Cref{sub:glosses}), you can automatise using glosses to a large degree.

\ea\label{ex:hungarian}Hungarian \parencite[adapted from][469]{EKiss2008}\\
    \gll    A \emph{pro}\textsubscript{i} diák-ja-i-t \emph{minden} \emph{tanár}\textsubscript{i} szeret-i.\\
            the {} students-\Poss.\Tsg-\Poss.\Pl-\Acc{} every teacher like-\Tsg.\Sbj>\Third.\Obj{}\\
    \glt    `Every teacher likes his/her students.'\\
\z

Examples with multiple parts are numbered as shown in~\eqref{ex:multiple}.
Syntax trees should be numbered like examples rather than like figures,
see~\eqref{ex:tree}, for example.

\ea\label{ex:multiple}Example with sub-examples\\
    \ea One
    \ex Two
    \z
\z

\ea\label{ex:tree}A simple structure\\
    \begin{forest}
        [XP
            [YP,name=yp ]
            [Xʹ
                [X,name=x ]  
                [ZP ]
            ]
        ]
        \draw [dashed, <->, >=Triangle, bend left=60]
            (x.200) to node [left, font=\small] {Agree} (yp.south west);
    \end{forest}
\z

\subsection{Tables and figures}

Tables and figures should also be numbered (e.g.\ \enquote{Table 3}) and they
should have captions below the table or figure. An example is shown
in~\Cref{tab:table}. Do not use vertical bars in tables
unless really necessary.

\begin{table}[ht]
    \centering
    \begin{tabular}{ll}
    \toprule
        Column 1 & Column 2 \\
    \midrule
        One & Two\\
        Three & Four\\
    \bottomrule
    \end{tabular}
    \caption{This is a table}
    \label{tab:table}
\end{table}

\section{Useful tips for the \LaTeX{} template}

\subsection{Title and authors}

You can specify the author(s) (and acknowledgements) as shown
in~\Cref{lst:author-single,lst:author-multiple}.

\begin{listing}[ht]
    \begin{minted}[linenos,bgcolor=gray!10!white,autogobble]{latex}
    \author{% 
        András Bárány \orcidlink{0000-0003-2356-4338}%
        \thanks{This is the place for acknowledgements.}\\
        University of Edinburgh\\
        \email{andras.barany@ed.ac.uk}
    }
    \end{minted}
    \caption{Single author}\label{lst:author-single}
\end{listing}

\begin{listing}[ht]
    \begin{minted}[linenos,bgcolor=gray!10!white,autogobble]{latex}
    \author{%
        AuthorFirst AuthorLast\\
        Affiliation\\
        \email{authorlast@affiliation.edu}
        \and
        Author2First Author2Last\\
        Affiliation\\
        \email{author2last@affiliation.edu}
    }
    \end{minted}
    \caption{Multiple authors}\label{lst:author-multiple}
\end{listing}

For the headers, please complete the section in the preamble surrounded by 
\enquote{\texttt{\%\% Specify paper metadata ====}} The settings for the
current file are shown in~\Cref{lst:metadata}.

\begin{listing}[ht]
    \begin{minted}[linenos,bgcolor=gray!10!white,autogobble]{latex}
    %% Specify paper metadata ====
    
    % full title of paper
    \newcommand{\fulltitle}{GLOW proceedings template}
    % short title of paper for header
    \newcommand{\shorttitle}{GLOW proceedings template}
    % last name of author(s); for multiple use X & Y; X, Y, & Z; X et al.
    \newcommand{\authorlast}{Bárány}
    
    %% Specify paper metadata ====
    \end{minted}
    \caption{Specifying title and headers}\label{lst:metadata}
\end{listing}

\subsection{Citations and references}\label{sec:bibliography}

This template uses the \texttt{biblatex} package. The most common citation
commands are the following:

\begin{itemize}
    \item In-text citation: \verb+\textcite{EKiss2008}+ prints \enquote{\textcite{EKiss2008}}
    \item Citation in parentheses: \verb+\parencite{EKiss2008}+ prints \enquote{\parencite{EKiss2008}}
    \item Citation without parentheses: \verb+\cite{EKiss2008}+ prints \enquote{\cite{EKiss2008}}
\end{itemize}

Each of these commands can be extended with optional arguments:

\begin{itemize}
    \item An optional argument immediately before the citation prints post-text:
    \begin{itemize}
        \item \verb+\textcite[469]{EKiss2008}+ prints \enquote{\textcite[469]{EKiss2008}}
        \item \verb+\parencite[469]{EKiss2008}+ prints \enquote{\parencite[469]{EKiss2008}}
    \end{itemize}
    \item A second optional argument preceding the other one prints pre-text (this doesn't work well with \verb+textcite+):
    \begin{itemize}
        \item \verb+\cite[cf.][469]{EKiss2008}+ prints \enquote{\cite[cf.][469]{EKiss2008}}
        \item \verb+\parencite[cf.][]{EKiss2008}+ prints \enquote{\parencite[cf.][]{EKiss2008}}
    \end{itemize}
\end{itemize}

You can add the bibliography to the paper using the commands shown
in~\Cref{lst:bibliography}.

\begin{listing}[h!t]
    \begin{minted}[linenos,bgcolor=gray!10!white,autogobble]{latex}
        \newrefcontext[sorting=nyt]
        \printbibliography
    \end{minted}
    \caption{Commands to add bibliography}\label{lst:bibliography}
\end{listing}

\subsection{Glosses and abbreviations}\label{sub:glosses}

The template loads the \texttt{leipzig} and \texttt{glossaries} packages. These can
be used to insert abbreviations in the text and to generate a list of abbreviations.

\begin{itemize}
    \item Typing \verb+\Acc{}+ prints \enquote{\Acc}
    \item Typing \verb+\Fsg{}+ prints \enquote{\Fsg}
\end{itemize}

You can specify your own abbreviations. If you add

\begin{minted}{latex}
\newacronym{wco}{WCO}{Weak crossover}
\end{minted}

to your preamble, you'll be able to use the command \verb+\gls+ to output the
abbreviation:

\begin{itemize}
    \item Typing \verb+\gls{wco}+ prints \enquote{\gls{wco}} (on first use!)
    \item When you type the same again, it prints \enquote{\gls{wco}}
    \item The abbreviation is added to the list of abbreviations
    (see~\Cref{lst:abbreviations})
\end{itemize}

\begin{listing}[ht]
    \begin{minted}[linenos,bgcolor=gray!10!white,autogobble]{latex}
        \section*{Abbreviations}
        \printglossary[style=inline, type=\leipzigtype]{}
    \end{minted}
    \caption{Commands to add list of abbreviations}\label{lst:abbreviations}
\end{listing}

\section{Questions, comments, \dots{}}

If you have any questions or comments, get in touch!

\section*{Abbreviations}
\printglossary[style=inline, type=\leipzigtype]{}

\section*{Acknowledgements}

This is the place for acknowledgements. In this case, I am grateful to Radek
Šimík, Lena Lohninger, and Susi Wurmbrand for comments!

\newrefcontext[sorting=nyt]
\printbibliography

\end{document}
