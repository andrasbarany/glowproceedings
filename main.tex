%TC:ignore
\documentclass[%
    fontsize=12pt,
    numbers=noenddot,
    paper=a4,
    toc=bib,
]{scrartcl}

% Language using polyglossia
\usepackage{polyglossia}
\setmainlanguage[variant=british,ordinalmonthday=false]{english}

% biblatex
\usepackage[%
    backend=biber,
    doi=true,
    language=british,
    maxbibnames=9,
    maxcitenames=2,
    mincrossrefs=3,
    natbib=true,
    sortcites,
    style=unified,
    useprefix=true,
    ]{biblatex}

\assignrefcontextentries[]{*}

% remove "p(p)."
\DeclareFieldFormat{postnote}{#1} % no postnote prefix in "normal" citation commands

% no shorthands

\DeclareFieldInputHandler{shorthand}{\def\NewValue{}}

% New citeposs command, more flexible

\usepackage{xifthen}

% use pubstate

\DeclareSourcemap{%
    \maps[datatype=bibtex]{%
        \map[overwrite=true]{% Notice the overwrite: replace one field with another
            \step[fieldsource=pubstate]
            \step[fieldset=year, origfieldval]
        }
    }
}

%% but don't print pubstate in the bibliography apart from the year

\AtEveryBibitem{%
    \clearfield{pubstate}%
}

%% add "to appear" as a proper pubstate, including capitalisation

\NewBibliographyString{toappear}
\DefineBibliographyStrings{english}{%
  toappear = {to appear},
}


%% Fonts, languages

\usepackage[warnings-off={mathtools-colon, mathtools-overbracket}]{unicode-math}

% kxetex
\usepackage{fontspec}
\defaultfontfeatures{Mapping=tex-text}

\usepackage[sb]{libertinus-otf}
\usepackage[scale=MatchLowercase]{FiraMono}

% Nicer tables
\usepackage{booktabs}
\usepackage[table]{xcolor}
\usepackage{colortbl}
\usepackage{microtype}
\usepackage{tabularx}

\usepackage[unicode,hidelinks]{hyperref}
\usepackage[normalem]{ulem} % for strikethrough \sout

%% captions (KOMA-script)
\setkomafont{caption}{\small}
\setkomafont{captionlabel}{\small\bfseries}
\renewcommand{\captionformat}{\quad}

%% Headers and footers
\usepackage{textcase}
\usepackage[automark,markcase=lower]{scrlayer-scrpage}
\pagestyle{scrheadings}
\setkomafont{pagenumber}{\footnotesize}%\sffamily}
\setkomafont{pageheadfoot}%
    {\small\addfontfeature{LetterSpace=10,Numbers=OldStyle}\scshape\sffamily}
\clearpairofpagestyles{}
\cfoot[\pagemark]{\pagemark}
\rohead{\MakeTextLowercase{\emph{\shorttitle}}}
\lohead{\MakeTextLowercase{\authorlast}}

%% Titles and headings (KOMA-Script)

% Titlepage
\setkomafont{author}{\large\sffamily}

% TOC
\setcounter{tocdepth}{2}
\setkomafont{partentry}{\large\sffamily\bfseries}
\KOMAoptions{headings=small}

% Chapter/section headings
\setcounter{secnumdepth}{3}

% Linguistics
\usepackage{langsci-gb4e}
\newcommand{\judgement}[1]{\makebox[0pt][r]{#1}}

%% glossary of abbreviations
\usepackage{glossaries}
\usepackage{glossary-inline}
\usepackage{leipzig}
\makeglossaries{}

\usepackage{cleveref}
\usepackage{enumitem}

%% Line breaks
\widowpenalty=10000
\clubpenalty=10000

% useful 
\newcommand{\email}[1]{\href{mailto:#1}{#1}}
\usepackage[autostyle=true,english=american]{csquotes}
\renewcommand*{\mkccitation}[1]{ (#1)}

% title page
\title{\fulltitle}
\date{}

%TC:endignore


% Add your bibliography here
\addbibresource{main.bib}

%% Specify paper metadata ----

% full title of paper
\newcommand{\fulltitle}{GLOW proceedings template}
% short title of paper for header
\newcommand{\shorttitle}{GLOW proceedings template}
% last name of author(s); for multiple use X & Y; X, Y, & Z; X et al.
\newcommand{\authorlast}{Bárány}

%% Specify paper metadata ----

% Specify title, author, etc.
\author{%
    András Bárány, Bielefeld University\\
    \email{andras.barany@uni-bielefeld.de}
}

% Specify your own packages here (keep this to a minimum!)
\usepackage{minted}
\newacronym{wco}{WCO}{weak crossover}
\usepackage{layout}

\begin{document}

\maketitle

\begin{abstract}
    \noindent\textbf{Abstract:} This style sheet illustrates the GLOW proceedings
    \LaTeX{} templates and describes how to use it.\\

    \noindent\textbf{Keywords:} \emph{keyword 1}, \emph{keyword 2}, \emph{keyword 3}
\end{abstract}

\section{This template}

If you are reading this file, you have downloaded the GLOW proceedings template
or you have accessed it on overleaf. This template consists of three files:

\begin{itemize}
    \item \texttt{glow-proceedings-preamble.tex}
    \item \texttt{main.tex}
    \item \texttt{main.bib}
\end{itemize}

You can ignore \texttt{glow-proceedings-preamble.tex} and simply work with
\texttt{main.tex} to write your paper and \texttt{main.bib} to include your
bibliography. If you require any additional packages, add them to
\texttt{main.tex} just before the beginning of the document.

This file describes the template and serves as a style sheet for GLOW proceedings.
Please follow the guidelines stated here otherwise we might not be able to accept
your submission.

\section{Style guidelines}

It's easiest to follow the style guidelines if you use the \LaTeX{} template:
you can download it or use it directly on overleaf. We very much welcome
submissions in \LaTeX{} as they make editing much easier.

Here are a few requirements for the proceedings:

\begin{itemize}
    \item Paper size: A4
    \item Typefaces (fonts): \textbf{Libertinus Serif} for text,
        \textbf{Libertinus Sans} for the title, author information, and
        headings\footnote{Why these typefaces? They are freely available
        open-source typefaces with very good support for Unicode characters.
        If you're using a recent version of \LaTeX{}, you'll have them
        installed already. If not, you can download them from
    \url{https://github.com/alerque/libertinus/releases}.}
    \item Each paper has an abstract (below the title)
    \item Each paper has up to five keywords (below the abstract)
    \item For citations and the bibliography, follow the Unified Style Sheet
        for Linguistics; if you use \LaTeX{}, you're set with this template
        (see~\Cref{sec:bibliography})
        \begin{itemize}
            \item If you're not using \LaTeX{}, format your references as in
                this document, get in touch, or consult
                \url{https://clas.wayne.edu/linguistics/resources/style}
        \end{itemize}
\end{itemize}

\ea Hungarian \parencite{EKiss2008}\\
    \gll Test\\
         test\\
    \glt \enquote{test}
\z

% \ea
%     $A = b$ a,b,c,d, $a,b,c,d,$
% \z
%
% \ea
%     \Acc{} á ʔ ø \textbf{å}
% \z
%
% \ea Semantic formula test
% \ea $\lambda P\lambda Q\forall x\big[P(x)\rightarrow \exists y[Q(x)(y)]\big]$
% \ex $\lambda Q\forall x\big[\textsc{glow member}(x)\rightarrow\exists y[Q(x)(y)]\big]$
% \z\z

\section{Title and authors}

\begin{verbatim}
\author{%
    András Bárány, Bielefeld University\\
    \email{andras.barany@uni-bielefeld.de}
}
\end{verbatim}

\begin{verbatim}
\author{%
    AuthorFirst AuthorLast, Affiliation\\
    \href{mailto:author@affiliation.edu}{authorlast@affiliation.edu}
    \and
    Author2First Author2Last, Affiliation\\
    \href{mailto:author@affiliation.edu}{author2last@affiliation.edu}
}
\end{verbatim}

\section{Citations and bibliography}\label{sec:bibliography}

This template uses the \texttt{biblatex} package. The most common citation
commands are the following:

\begin{itemize}
    \item In-text citation: \verb+\textcite{EKiss2008}+ prints \enquote{\textcite{EKiss2008}}
    \item Citation in parentheses: \verb+\parencite{EKiss2008}+ prints \enquote{\parencite{EKiss2008}}
    \item Citation without parentheses: \verb+\cite{EKiss2008}+ prints \enquote{\cite{EKiss2008}}
\end{itemize}

Each of these commands can be extended with optional arguments:

\begin{itemize}
    \item An optional argument immediately before the citation prints post-text:
    \begin{itemize}
        \item \verb+\textcite[469]{EKiss2008}+ prints \enquote{\textcite[469]{EKiss2008}}
        \item \verb+\parencite[469]{EKiss2008}+ prints \enquote{\parencite[469]{EKiss2008}}
    \end{itemize}
    \item A second optional argument preceding the other one prints pre-text (this doesn't work well with \verb+textcite+):
    \begin{itemize}
        \item \verb+\cite[cf.][469]{EKiss2008}+ prints \enquote{\cite[cf.][469]{EKiss2008}}
        \item \verb+\parencite[cf.][]{EKiss2008}+ prints \enquote{\parencite[cf.][]{EKiss2008}}
    \end{itemize}
\end{itemize}

You can add the bibliography to the paper using the commands shown in~\Cref{lst:bibliography}.

\begin{listing}[H]
    \begin{minted}[linenos,bgcolor=gray!10!white,autogobble]{latex}
        \newrefcontext[sorting=nyt]
        \printbibliography
    \end{minted}
    \caption{Commands to add bibliography}\label{lst:bibliography}
\end{listing}

\section{Glosses and abbreviations}

The template loads the \texttt{leipzig} and \texttt{glossaries} packages. These can
be used to insert abbreviations in the text and to generate a list of abbreviations.

\begin{itemize}
    \item Typing \verb+\Acc{}+ prints \enquote{\Acc}
    \item Typing \verb+\Fsg{}+ prints \enquote{\Fsg}
\end{itemize}

You can specify your own abbreviations. If you add

\begin{minted}{latex}
\newacronym{wco}{WCO}{Weak crossover}
\end{minted}

to your preamble, you'll be able to use the command \verb+\gls+ to output the
abbreviation:

\begin{itemize}
    \item Typing \verb+\gls{wco}+ prints \enquote{\gls{wco}} (on first use!)
    \item When you type the same again, it prints \enquote{\gls{wco}}
    \item The abbreviation is added to the list of abbreviations
    (see~\Cref{lst:abbreviations})
\end{itemize}

\begin{listing}[H]
    \begin{minted}[linenos,bgcolor=gray!10!white,autogobble]{latex}
        \section*{Abbreviations}
        \printglossary[style=inline, type=\leipzigtype]{}
    \end{minted}
    \caption{Commands to add list of abbreviations}\label{lst:abbreviations}
\end{listing}

\section*{Abbreviations}
\printglossary[style=inline, type=\leipzigtype]{}

\newrefcontext[sorting=nyt]
\printbibliography

% \pagebreak
%
% \layout

\end{document}
